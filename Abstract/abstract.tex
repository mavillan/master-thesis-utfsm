

\section*{Resumen}

La reproducibilidad experimental es la capacidad de volver a ejecutar un experimento con o sin la introducción de cambios en él y que los resultados sean consistentes con los originales. 
Para permitir la reproducibilidad, la comunidad científica ha animado a los investigadores a conservar las descripciones de estos experimentos. 
Sin embargo, el trabajo en la descripción de la infraestructura no ha resuelto de forma completa.
En este trabajo proponemos un sistema para describir automáticamente los entornos computacionales utilizados en los experimentos computacionales. 
Para ello, proponemos utilizar la virtualización basada en contenedores para distribuir los experimentos a través de imágenes de software y un sistema de anotación que permita describir estas imágenes de software. 
Las imágenes son una versión mínima de un sistema operativo (contenedor) que permite el despliegue de múltiples paquetes de software aislados dentro de él. 
Proponemos el uso de Contenedores Docker para la conservación del entorno científico y un framework para capturar el valioso conocimiento sobre los recursos de un experimento computacional que se ejecuta en un contenedor Docker.
Como resultado, nos enfrentamos al desafío lógico y físico de la conservación. 
 \newpage
\section*{Abstract}
% 
Experiment reproducibility is the ability to re-run an experiment with or without introducing changes to it and getting results that are consistent with the original ones. 
To allow reproducibility, the scientific community encourages researchers to publish descriptions of these experiments. 
However, the recommendations for obtaining a description of the infrastructure has not resolved completely.
In this paper, we propose a system to describe computational environments used in computer experiments automatically. We propose to use container-based virtualization for distributing software experiments throughout software images and an annotation system that will allow describing these software images. The images are a minimal version of an OS (container) that allow the deployment of multiple isolated software packages within it. 
We propose the usage of Docker for the conservation of the scientific environment and a framework to capture the valuable knowledge about the resources of a computational experiment running on a Docker Container. As a result, we are facing the logical and physical conservation challenge. 