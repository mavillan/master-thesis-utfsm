\chapter{Introducción}
\label{Chapter1} 
\newcommand{\keyword}[1]{\textbf{#1}}
\newcommand{\tabhead}[1]{\textbf{#1}}
\newcommand{\code}[1]{\texttt{#1}}
\newcommand{\file}[1]{\texttt{\bfseries#1}}
\newcommand{\option}[1]{\texttt{\itshape#1}}
La reproducibilidad experimental es la capacidad de volver a ejecutar un experimento con o sin la introducción de cambios en él y la obtención de resultados que sean consistentes con los originales.
La introducción de cambios permite evaluar diferentes características de ese experimento ya que los investigadores pueden modificarlo gradualmente, mejorando y re-orientando los métodos y condiciones experimentales~\cite{stodden2010reproducible}.
Investigadores han argumentado sobre la importancia de reproducir experimentos científicos debido a la necesidad de corroborar los resultados o de contribuir en el mismo trabajo.
Sin embargo, el proceso de reproducibilidad puede ser tan complejo y costoso que ha sido referido como una tarea del tipo forense \cite{baggerly2009deriving}.
De hecho, estudios controversiales y escándalos han expuesto la necesidad de mejorar los procesos de reproducibilidad de las investigaciones. 
Algunos ejemplos son: \cite{ioannidis2009repeatability} donde se estudia la reproducibilidad del múltiples estudios del área biología afirmando que comúnmente no se disponen los datos públicamente, e incluso cuando están disponibles, se desconoce si los resultados publicados son reproducibles o estudios en área de la computación donde se muestra la dificultades de reproducir 402 artículos de conferencias de la \emph{Association for Computing Machinery (ACM)}\cite{collberg2015repeatability}.

Para permitir la reproducibilidad del experimento es necesario proporcionar suficiente información sobre él mismo, permitiendo comprenderlo, evaluarlo y volver a construirlo. 
Normalmente, los experimentos se describen en flujos de trabajo científicos o \emph{workflows} (representaciones que permiten gestionar cálculos a gran escala) que puede ser ejecutados en sistemas informáticos distribuidos.
%Aparece el problema de la conservación de los experimentos
Para permitir la reproducibilidad de estos workflows científicos es necesario, en primer lugar, abordar un problema de conservación del experimento.
La conservación es un acto de proporcionar información compresible que describa el contexto original del experimento. Por lo tanto, se debe garantizar que exista la suficiente información sobre los experimentos para que sea posible construirlos nuevamente por un tercero, replicando sus resultados sin ninguna información adicional del autor original ~\cite{garijo2013quantifying}.
%Cual es la conversación que se hace ahora
Para alcanzar la conservación, la comunidad se ha enfocado en la conservación de las ejecuciones del experimento conservando los datos, código y la descripción, pero dejando de lado la infraestructura subyacente (e.g, recursos computacionales y componentes de software).
Existen enfoques que se han enfocado en conservar el ambiente computacional de un experimento como WICUS~\cite{santana2017reproducibility} que propone vocabularios semánticos para la conservación o el proyecto TIMBUS~\footnote{\url{http://www.timbusproject.net/}}~\cite{dappert2013describing} que describe los procesos de negocios y el software y el hardware subyacente. 
%Hablamos los tipos de conservación que existe
Los autores en~\cite{santana2017reproducibility} identificaron dos enfoques para conservar el ambiente computacional de un experimento científico: la conservación física, en la que los objetos de investigación dentro del experimento se conservan en un entorno virtual (e.g., máquinas virtuales); y la conservación lógica, en la que se describen las principales capacidades de los recursos del ambiente computacional utilizando vocabularios semánticos para que el investigador pueda reproducir un entorno equivalente.
Para ello, definieron un proceso para documentar la aplicación de workflow, su sistema de gestión relacionado y las dependencias de ambos.
Sin embargo, este proceso es manual, dejando un trabajo complejo para los científicos. 
Por otro lado, los autores no enfrentan la conservación física del entorno computacional argumentando que los costos necesarios para almacenar los ambientes computacionales son muy altos, la existencia de dificultades dado a las restricciones políticas de las organizaciones a cargo ó procesos de decaimiento del ambiente computacional en el tiempo~\cite{DBLP:journals/fgcs/DeelmanVJRCMMCS15}. 
Sin embargo, la conservación física simplifica la ejecución del experimento a cargo de nuevo investigador. Dado que evita la compleja tarea compleja del el proceso de reconstrucción del ambiente.
El problema de la reproducibilidad no sólo sucede en la comunidad científica. También compañías tecnológicas enfrenta problemas similares cuando ellos desean distribuir cualquier producto de software a múltiples servidores o plataformas.
Para resolver esto, las compañías utilizan la virtualización. Esta tecnología se refiere a una característica que permite la existencia de múltiples instancias aisladaso llamadas máquinas virtuales o contenedores.
Una de las tecnologías de virtualización más populares en la industria es Docker\footnote{\url{https://www.docker.com/}}. 
Esta herramienta implementa la virtualización de software mediante la creación de versiones mínimas de un sistema operativo base llamado contenedor
Los contenedores Docker pueden ser vistos como máquinas virtuales ligeras que permiten el ensamblaje de un entorno computacional, incluyendo todas las dependencias necesarias, como bibliotecas, configuración, código, datos, entre otros.
%need
Para reproducir experimentos científicos computacionales de otros investigadores, es obligatorio permitir que los científicos compartan estos experimentos y se conserve el ambiente computación con el objetivo de reproducir el ambiente computacional  (el mismo o muy similar).
De esta manera, un científico tendrá garantías de que el experimento se está ejecutando en el mismo entorno o similar en comparación al entorno original.
Por lo tanto, se necesita un procedimiento que garantice ambos requisitos: preservar los entornos tanto lógicos como físicos para volver a ejecutar los flujos de trabajo de datos con garantías de reproducibilidad.
Para permitir la conservación lógica es necesario proporcionar una manera de describir el entorno computacional en el que se ejecutó el experimento. 
La conservación lógica permite describir el entorno computacional en el que se ejecutó el experimento, por lo tanto los investigadores pueden conocer los componentes de software y al volver a construir el ambiente computacional del experimento en cualquier momento y conocer cuáles los cambios que sean producido a nivel de los recursos.
Considerando lo anterior, se propone una solución para mejorar la conservación física y lógica mediante el uso de contenedores, y un proceso automático que describa los requerimientos del ambiente computacional y luego sean almacenados utilizando modelos semánticos. Permitiendo, al investigador conocer los componentes del ambiente computacional de su experimento.
Para ello, se propone en primer lugar utilizar las imágenes Docker como medio para preservar el entorno físico de un experimento. Se utilizarán contenedores ya que son ligeros, de fácil adopción y lo que es más importante, son más fáciles de describir automáticamente, por lo que se mejora el proceso de documentación de los workflows científicos.
Con Docker, los usuarios pueden distribuir estos entornos computacionales a través de imágenes de software utilizando un repositorio público llamado DockerHub\footnote{\url{https://hub.docker.com/}}. Esto permite alcanzar la conservación física gracias a los repositorios y las imágenes Docker.
Con el fin de lograr una conservación lógica, se construye un sistema anotador para las imágenes Docker que describe el sistema de gestión del workflow y las dependencias asociados. De esta manera, se pretende abordar la conservación lógica.
Para validar la solución se reproducen cinco experimentos computacionales diferentes. Estos experimentos abarcan diferentes sistemas de workflows, lenguajes y configuraciones, lo que se demuestra que el enfoque es genérico y puede aplicarse a múltiples experimentos computacionales.
Por ende, se construyen los ambientes computacionales de estos experimentos, se describen lógicamente, se utiliza estas descripciones para la resolución de problemas de ejecución y se reproduce el ambiente computacional en base a las descripciones lógicas que se obtuvieron anteriormente.
Para validar el enfoque, se comparan los valores o estructuras de los resultados de los experimentos y registro de eventos de información o errores del sistema de workflow.

El presente documento está estructurado de la siguiente manera. 
En el Capítulo 2 se presenta un estado del arte sobre la conservación y reproducibilidad de experimentos computacionales y la descripción de las herramientas que serán utilizadas en el enfoque propuesto.
En el Capítulo 3, se discuten los problemas abiertos en la reproducibilidad de ambientes computacionales, los objetivos e hipótesis del presente trabajo.
Luego, en el Capítulo 4, se presentará el sistema de anotación para obtener las descripciones y la ontología propuesta para almacenarla.
En el Capítulo 5, se aplicará el enfoque a diversos experimentos reales y se mostrarán los resultados y problemas encontrados. 
Finalmente, en Capítulo 6 se señalan las conclusiones obtenidas en este trabajo y se plantean las proyecciones para un trabajo futuro.